\chapter{Chapter 3}

This chapter showcases including figures, listings and tables.

\section{Figures}

Figures can be created by using the figure keyword in the \lstinline{verb|begin|} statement.
Every figure needs a caption, so that it appears in the figure listing and for a reasonable description.

A label can be used to reference the figure later on in text.

\begin{figure}[H]
    \centering
    \includegraphics[height=2cm]{images/DHBW-Logo.png}
    \caption{My Custom Caption for my Figure}
    \label{fig:placeholder}
\end{figure}

As a label with value \lstinline{fig:placeholder} has been defined, it can be referenced by using the \lstinline{\ref} command. This way you can easily reference \ref{fig:placeholder} in your text.

\section{Listings}

Listings allow you to reference/show code in your report.

\subsection{Inline Listing}

If you want to quickly reference a variable, value, etc., you can use inline listings to do so by using \lstinline{\lstinline}.

\subsection{Listings}

If you want to reference big code snippets, you can use \lstinline{lstlisting} with the correct options.
Please make sure, that the language you're using, has syntax highlighting settings in the \textit{preamble}.
Do not forget to define a caption for it to appear in the listings sections. You can also define a label for later referencing, like \ref{lst:helloworld}.

\begin{lstlisting}[language=JavaScript, label=lst:helloworld, caption=A Hello World sample]
const bMyBool = true;
if (bMyBool) {
    console.log("Hello World");
}
\end{lstlisting}

\section{Tables}

You can define tables to showcase tabular structures in your report.

\begin{table}[htb]
    \centering
    \begin{tabular}{l|c} % this defines the alingment of each column
        Column 1 (left-aligned) & Column 2 (centered) \\
        \hline
        \texttt{Row 1} & 20 \\
        \texttt{Row 2} & 20 \\
        \hline
        \hline
        Sum & 40 \\
        Average & 20
    \end{tabular}
    \caption{Another caption for it to show up in the table section}
\end{table}