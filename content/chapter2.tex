\chapter{Chapter 2}

This chapter will showcase using acronyms and citing.

\section{Acronyms}

Use acronyms by using the \lstinline{\ac} command. Define acronyms with \lstinline{\acro} in the \textit{abbreviations.tex} file.

First occurrences of the acronym will be written out, while subsequent occurrences will be abbreviated.

\ac{API} is written in long-form here. \ac{API} is abbreviated here.

\section{Citing}

You can cite by using the \lstinline{\cite} command. To do so, you need to define sources in the \textit{sources.bib} file. These sources need to have unique identifiers for referencing.

In \textit{sources.bib} we've defined a source referencing a BibTex Guide with identifier \textbf{BibTex}. You can reference this by using \lstinline{\cite} \cite{BibTex}.
\\

\subsection{Inline Text Quote}

You can use \lstinline{\textquote} for inline quoting. Per \cite{BibTex}, \enquote{BibTeX is an indispensable ally. Despite the intricate details surrounding LaTeX packages, citation styles, and formatting, with a structured approach, BibTeX can be easy to grasp.}

\subsection{Block Quotes}

You can use \lstinline{\blockquote} to quote in a block:
\begin{quote}
    For those diving into academic or technical writing, BibTeX is an indispensable ally. Despite the intricate details surrounding LaTeX packages, citation styles, and formatting, with a structured approach, BibTeX can be easy to grasp.\cite{BibTex}
\end{quote}

See \texttt{csquotes} for more information on how to quote. \cite{csquotes}
