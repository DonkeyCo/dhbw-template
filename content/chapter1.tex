% ===================================================================
% CHAPTER 1: GETTING STARTED WITH LATEX
% ===================================================================
% 
% This chapter demonstrates basic LaTeX features and serves as a learning
% guide for beginners. It shows how to create chapters, sections, and
% various types of content.
%
% For beginners: This chapter is both an example and a tutorial. You can
% replace this content with your actual report content, but keep it as
% reference while learning LaTeX.
% ===================================================================

\chapter{Getting Started with LaTeX}
\label{chap:getting-started}

% ===================================================================
% CHAPTER INTRODUCTION
% ===================================================================
% Always start chapters with a brief introduction explaining what will be covered

This chapter serves as both an introduction to your report and a demonstration of basic LaTeX features. It showcases how to create structured content including chapters, sections, subsections, and various formatting elements that you'll commonly use in academic writing.

\important{Note}: You can replace this content with your actual report content once you're comfortable with LaTeX syntax.

% ===================================================================
% BASIC SECTIONS AND STRUCTURE
% ===================================================================

\section{Document Structure and Sections}
\label{sec:structure}

LaTeX documents are hierarchically structured using different sectioning commands. This creates a logical flow and enables automatic numbering and table of contents generation.

Here are the main structural elements you can use:

\begin{itemize}
    \item \code{\textbackslash chapter\{Title\}} - Top-level divisions (like this chapter)
    \item \code{\textbackslash section\{Title\}} - Main sections within chapters
    \item \code{\textbackslash subsection\{Title\}} - Subdivisions within sections
    \item \code{\textbackslash subsubsection\{Title\}} - Further subdivisions (rarely needed)
\end{itemize}

\subsection{Cross-References and Labels}
\label{subsec:references}

One of LaTeX's most powerful features is automatic cross-referencing. You can refer to any numbered element (chapters, sections, figures, tables, equations) anywhere in your document.

For example:
\begin{itemize}
    \item This is \secref{sec:structure} of \chapref{chap:getting-started}
    \item We're currently in \secref{subsec:references}
    \item Later we'll see \figref{fig:example-image} and \tabref{tab:example-table}
\end{itemize}

\note{Labels must be unique throughout your document. Use descriptive prefixes like \code{chap:}, \code{sec:}, \code{fig:}, \code{tab:} to organize them.}

\subsection{Text Formatting Examples}

LaTeX provides various ways to format text for emphasis and clarity:

\begin{itemize}
    \item \textbf{Bold text} using \code{\textbackslash textbf\{text\}}
    \item \textit{Italic text} using \code{\textbackslash textit\{text\}}
    \item \texttt{Monospace text} using \code{\textbackslash texttt\{text\}}
    \item \underline{Underlined text} using \code{\textbackslash underline\{text\}}
    \item \important{Important concepts} using our custom \code{\textbackslash important\{text\}} command
\end{itemize}

% ===================================================================
% LISTS AND ENUMERATIONS
% ===================================================================

\section{Lists and Enumerations}
\label{sec:lists}

Lists are essential for organizing information clearly. LaTeX provides several list types:

\subsection{Unordered Lists (Bullet Points)}

Use \code{itemize} environment for bullet points:

\begin{itemize}
    \item First item
    \item Second item with a longer explanation that might wrap to multiple lines
    \item Third item
    \begin{itemize}
        \item Nested item
        \item Another nested item
    \end{itemize}
    \item Fourth item
\end{itemize}

\subsection{Ordered Lists (Numbered)}

Use \code{enumerate} environment for numbered lists:

\begin{enumerate}
    \item First step in your process
    \item Second step with detailed explanation
    \item Third step
    \begin{enumerate}
        \item Sub-step A
        \item Sub-step B
    \end{enumerate}
    \item Final step
\end{enumerate}

\subsection{Description Lists}

Use \code{description} environment for term definitions:

\begin{description}
    \item[LaTeX] A document preparation system for high-quality typesetting
    \item[DHBW] Duale Hochschule Baden-Württemberg (Baden-Württemberg Cooperative State University)
    \item[PDF] Portable Document Format, the output of LaTeX compilation
\end{description}

% ===================================================================
% FIGURES AND IMAGES
% ===================================================================

\section{Including Images and Figures}
\label{sec:figures}

Images are essential for technical reports. LaTeX handles figure placement and numbering automatically.

\subsection{Basic Figure Example}

Here's how to include an image with proper captioning and labeling:

\begin{figure}[htbp]
    \centering
    \includegraphics[width=0.6\textwidth]{images/logo-placeholder.jpg}
    \caption{Example of including an image with caption and automatic numbering}
    \label{fig:example-image}
\end{figure}

As you can see in \figref{fig:example-image}, the image is automatically numbered and can be referenced throughout your document.

\subsection{Figure Placement Options}

The \code{[htbp]} options control where LaTeX places your figure:
\begin{itemize}
    \item \code{h} - Here (approximately where you put it in the text)
    \item \code{t} - Top of page
    \item \code{b} - Bottom of page  
    \item \code{p} - On a separate page of floats
\end{itemize}

\warning{LaTeX may not place figures exactly where you expect. This is normal behavior - LaTeX optimizes placement for the best overall document layout.}

% ===================================================================
% TABLES
% ===================================================================

\section{Creating Tables}
\label{sec:tables}

Tables are crucial for presenting structured data. Here's an example of a well-formatted table:

\begin{table}[htbp]
    \centering
    \caption{Example table showing different data types and formatting}
    \label{tab:example-table}
    \begin{tabular}{@{}lcr@{}}
        \toprule
        \textbf{Item} & \textbf{Quantity} & \textbf{Price (€)} \\
        \midrule
        Laptops       & 15                & 899.99 \\
        Mice          & 25                & 29.95  \\
        Keyboards     & 20                & 79.50  \\
        \midrule
        \textbf{Total} & \textbf{60}      & \textbf{1,009.44} \\
        \bottomrule
    \end{tabular}
\end{table}

Key features of good table formatting:
\begin{itemize}
    \item Use \code{\textbackslash toprule}, \code{\textbackslash midrule}, and \code{\textbackslash bottomrule} for professional lines
    \item Center the table with \code{\textbackslash centering}
    \item Align columns appropriately (l=left, c=center, r=right)
    \item Always include a descriptive caption
    \item Reference tables in your text: "As shown in \tabref{tab:example-table}..."
\end{itemize}

% ===================================================================
% CODE LISTINGS
% ===================================================================

\section{Including Source Code}
\label{sec:code}

For technical reports, you often need to include source code. LaTeX provides excellent code formatting capabilities.

\subsection{Inline Code}

Use the \code{\textbackslash code\{\}} command for short code snippets like \code{function\_name()} or \code{variable\_x} within sentences.

\subsection{Code Blocks}

For longer code examples, use the \code{lstlisting} environment:

\begin{lstlisting}[language=JavaScript, caption=Example JavaScript function, label=lst:example-js]
function calculateSum(numbers) {
    // Initialize sum variable
    let sum = 0;
    
    // Iterate through array and add each number
    for (let i = 0; i < numbers.length; i++) {
        sum += numbers[i];
    }
    
    return sum;
}

// Usage example
const myNumbers = [1, 2, 3, 4, 5];
const result = calculateSum(myNumbers);
console.log("Sum:", result); // Output: Sum: 15
\end{lstlisting}

The code in Listing~\ref{lst:example-js} demonstrates proper JavaScript syntax highlighting and formatting.

% ===================================================================
% MATHEMATICAL EXPRESSIONS
% ===================================================================

\section{Mathematical Expressions}
\label{sec:math}

LaTeX excels at typesetting mathematical content. Here are some examples:

\subsection{Inline Mathematics}

You can include mathematical expressions in your text, such as $E = mc^2$ or $\pi \approx 3.14159$.

\subsection{Display Mathematics}

For more complex equations, use display mode:

$$
f(x) = \sum_{i=0}^{n} \frac{a_i x^i}{i!} = a_0 + a_1 x + \frac{a_2 x^2}{2!} + \frac{a_3 x^3}{3!} + \ldots
$$

\subsection{Numbered Equations}

For important equations that you want to reference:

\begin{equation}
\label{eq:quadratic-formula}
x = \frac{-b \pm \sqrt{b^2 - 4ac}}{2a}
\end{equation}

Equation~\ref{eq:quadratic-formula} shows the famous quadratic formula.

% ===================================================================
% CITATIONS AND REFERENCES
% ===================================================================

\section{Citations and Bibliography}
\label{sec:citations}

Academic writing requires proper citations. This template uses modern biblatex for citation management.

\subsection{How Citations Work}

Add your references to \filepath{appendix/sources.bib}, then cite them in your text. The bibliography at the end of your document will be automatically generated.

\subsection{Citation Examples}

Here are different ways to cite sources:
\begin{itemize}
    \item Basic citation: LaTeX is a powerful typesetting system~\cite{lamport1994latex}.
    \item Citation with page numbers: Advanced LaTeX techniques are covered extensively~\cite[pp.~123-145]{companion2004latex}.
    \item Multiple citations: Several sources discuss this topic~\cite{knuth1984texbook,mittelbach2004latex}.
\end{itemize}

\note{Add your actual references to \filepath{appendix/sources.bib} and replace these example citations with your real sources.}

% ===================================================================
% CHAPTER SUMMARY
% ===================================================================

\section{Chapter Summary}
\label{sec:summary}

This chapter has introduced you to the fundamental LaTeX concepts you'll need for your report:

\begin{enumerate}
    \item Document structure with chapters, sections, and subsections
    \item Text formatting and emphasis
    \item Lists and enumerations  
    \item Figures and images
    \item Tables and data presentation
    \item Source code inclusion
    \item Mathematical expressions
    \item Citations and references
\end{enumerate}

In the following chapters, you can replace this example content with your actual report material, using these formatting techniques to create a professional, well-structured document.

\warning{Remember to update \filepath{config/metadata.tex} with your personal information before compiling your final document!}
