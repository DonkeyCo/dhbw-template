% ===================================================================
% CUSTOM COMMANDS AND SHORTCUTS
% ===================================================================
% 
% This file is where you can define your own LaTeX commands to make
% writing your document easier and more consistent.
%
% For beginners: Custom commands are like shortcuts or macros. Instead of
% typing the same complex LaTeX code repeatedly, you can create a simple
% command that does it for you.
%
% BENEFITS OF CUSTOM COMMANDS:
% - Save time by creating shortcuts for frequently used text/formatting
% - Ensure consistency throughout your document
% - Make changes easier (change the command definition to update everywhere)
% - Reduce typos and formatting errors
% ===================================================================

% ===================================================================
% TEXT SHORTCUTS AND ABBREVIATIONS
% ===================================================================
% Create shortcuts for commonly used phrases and abbreviations

% Example: Shortcut for "e.g." (exempli gratia)
\newcommand{\eg}{e.g.}                  % Usage: \eg instead of typing "e.g."

% Example: Shortcut for "i.e." (id est)  
\newcommand{\ie}{i.e.}                  % Usage: \ie instead of typing "i.e."

% Example: Shortcut for "etc." (et cetera)
\newcommand{\etc}{etc.}                 % Usage: \etc instead of typing "etc."

% Example: Company or project name (update throughout document easily)
% \newcommand{\companyname}{Your Company Name}    % Uncomment and customize
% \newcommand{\projectname}{Your Project Name}    % Uncomment and customize

% ===================================================================
% FORMATTING SHORTCUTS
% ===================================================================
% Commands for consistent formatting of special text

% Example: Highlight important terms (you can change the formatting easily)
\newcommand{\important}[1]{\textbf{\textit{#1}}}    % Usage: \important{key concept}

% Example: Format code/technical terms consistently
\newcommand{\code}[1]{\texttt{#1}}                  % Usage: \code{function_name}

% Example: Format file paths consistently
\newcommand{\filepath}[1]{\texttt{#1}}              % Usage: \filepath{/path/to/file}

% Example: Format menu items or UI elements
\newcommand{\menuitem}[1]{\textbf{#1}}              % Usage: \menuitem{File > Save}

% ===================================================================
% ACADEMIC AND TECHNICAL SHORTCUTS
% ===================================================================
% Commands specific to academic writing and technical documents

% Example: Reference figures consistently
\newcommand{\figref}[1]{Figure~\ref{#1}}            % Usage: \figref{fig:myimage}

% Example: Reference tables consistently  
\newcommand{\tabref}[1]{Table~\ref{#1}}             % Usage: \tabref{tab:results}

% Example: Reference chapters/sections consistently
\newcommand{\chapref}[1]{Chapter~\ref{#1}}          % Usage: \chapref{chap:intro}
\newcommand{\secref}[1]{Section~\ref{#1}}           % Usage: \secref{sec:method}

% Example: Consistent degree symbol
\newcommand{\degree}{\ensuremath{^\circ}}           % Usage: 45\degree

% ===================================================================
% MATHEMATICAL SHORTCUTS
% ===================================================================
% Commands for mathematical expressions (if your document contains math)

% Example: Common mathematical sets
% \newcommand{\R}{\mathbb{R}}              % Real numbers: \R
% \newcommand{\N}{\mathbb{N}}              % Natural numbers: \N
% \newcommand{\Z}{\mathbb{Z}}              % Integers: \Z

% Example: Common functions
% \newcommand{\abs}[1]{\left|#1\right|}   % Absolute value: \abs{x}

% ===================================================================
% MULTI-PARAMETER COMMANDS
% ===================================================================
% More complex commands that take multiple inputs

% Example: Consistent citation with page numbers
\newcommand{\citepp}[2]{\cite[pp.~#2]{#1}}          % Usage: \citepp{author2023}{15-20}

% Example: Inline code with language specification
\newcommand{\inlinecode}[2]{\lstinline[language=#1]|#2|}  % Usage: \inlinecode{Java}{System.out.println()}

% ===================================================================
% ENVIRONMENT SHORTCUTS
% ===================================================================
% Custom environments for special content

% Example: Create a "note" box for important information
\newcommand{\note}[1]{%
    \begin{mdframed}[backgroundcolor=yellow!20, linecolor=orange, linewidth=2pt]
    \textbf{Note:} #1
    \end{mdframed}
}

% Example: Create a "warning" box for critical information
\newcommand{\warning}[1]{%
    \begin{mdframed}[backgroundcolor=red!10, linecolor=red, linewidth=2pt]
    \textbf{Warning:} #1
    \end{mdframed}
}

% ===================================================================
% HOW TO CREATE YOUR OWN COMMANDS
% ===================================================================
%
% BASIC SYNTAX:
% \newcommand{\commandname}[number_of_parameters]{what_it_does}
%
% EXAMPLES:
% 
% 1. Simple text replacement (no parameters):
%    \newcommand{\myname}{John Doe}
%    Usage: Hello, I am \myname
%
% 2. Command with one parameter:
%    \newcommand{\highlight}[1]{\textbf{\textcolor{red}{#1}}}
%    Usage: \highlight{important text}
%
% 3. Command with multiple parameters:
%    \newcommand{\fullcite}[3]{\cite{#1}, #2, pp. #3}
%    Usage: \fullcite{author2023}{Journal Name}{15-20}
%
% 4. Optional parameters (advanced):
%    \newcommand{\mycommand}[2][default]{\textbf{#1}: #2}
%    Usage: \mycommand{text} or \mycommand[custom]{text}
%
% BEST PRACTICES:
% - Use descriptive command names
% - Test your commands before using them extensively
% - Document complex commands with comments
% - Keep commands simple and focused on one task
% - Use consistent naming conventions
%
% ===================================================================

% ===================================================================
% YOUR CUSTOM COMMANDS GO HERE
% ===================================================================
% Add your own custom commands below this line
% Remember to test them before using them in your document!

% Example template for your commands:
% \newcommand{\yourcommand}[1]{your LaTeX code here with #1}
