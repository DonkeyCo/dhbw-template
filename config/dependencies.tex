% ===================================================================
% LATEX PACKAGES AND DEPENDENCIES
% ===================================================================
% 
% This file loads all the LaTeX packages needed for your document.
% Packages extend LaTeX's functionality and provide additional features.
%
% For beginners: Think of packages like plugins or extensions that add
% specific capabilities to your document (like handling images, tables,
% citations, etc.).
%
% IMPORTANT: The order of package loading can matter! Some packages
% must be loaded before others to avoid conflicts.
% ===================================================================

% ===================================================================
% BASIC TEXT AND LANGUAGE SUPPORT
% ===================================================================
% These packages handle character encoding and language settings

\usepackage[utf8]{inputenc}     % Handle Unicode characters (ä, ö, ü, é, etc.)
\usepackage[english]{babel}     % English language support (hyphenation, etc.)
                                % Change to [german] if writing in German
\usepackage[babel]{csquotes}    % Context-sensitive quotation marks
\usepackage[T1]{fontenc}        % Better font encoding for European characters

% ===================================================================
% GRAPHICS AND FIGURES
% ===================================================================
% Packages for handling images and visual elements

\usepackage{graphicx}           % Include images (PNG, JPG, PDF)
\usepackage{wrapfig}            % Wrap text around figures
\usepackage{svg}                % Support for SVG vector graphics
\usepackage{xcolor}             % Define and use colors

% ===================================================================
% DOCUMENT FORMATTING AND LAYOUT
% ===================================================================
% Packages that control how your document looks

\usepackage{setspace}           % Control line spacing (single, double, 1.5x)
\usepackage{titletoc}           % Customize table of contents appearance

% ===================================================================
% CONDITIONAL LOGIC AND PROGRAMMING
% ===================================================================
% Packages for advanced document logic

\usepackage{ifthen}             % If-then-else logic in LaTeX
                                % Used for conditional content (confidential notices, etc.)

% ===================================================================
% SPECIALIZED CONTENT TYPES
% ===================================================================
% Packages for specific types of content

\usepackage{acronym}            % Manage abbreviations and acronyms
\usepackage{appendix}           % Better appendix handling

% ===================================================================
% HYPERLINKS AND NAVIGATION
% ===================================================================
% Make your PDF interactive with clickable links

\usepackage[                    % Clickable links (includes "nameref", "url" packages)
    hidelinks,                  % Hide the colored boxes around links
    breaklinks=true             % Allow URLs to break across lines
]{hyperref}

% ===================================================================
% SOURCE CODE DISPLAY
% ===================================================================
% Packages for showing programming code in your document

\usepackage{listings}           % Display source code with syntax highlighting
\usepackage{algorithm}          % Display algorithms
\usepackage[noend]{algpseudocode} % Write pseudocode for algorithms

% ===================================================================
% TABLES AND DATA PRESENTATION
% ===================================================================
% Packages for creating professional tables

\usepackage{booktabs}           % Professional table formatting (better lines)
\usepackage{multirow}           % Cells that span multiple rows
\usepackage{siunitx}            % Proper formatting of numbers and units
\usepackage{tabularx}           % Tables that automatically fit page width

% ===================================================================
% MATHEMATICAL CONTENT AND BOXES
% ===================================================================
% Packages for math and highlighted content boxes

\usepackage{amsthm}             % Mathematical theorems and proofs
\usepackage[framemethod=tikz]{mdframed} % Create colored boxes around content

% ===================================================================
% FONTS AND TYPOGRAPHY
% ===================================================================
% Packages for better font handling

\usepackage{sourcecodepro}      % Source Code Pro font for code listings
                                % Provides a clean, professional monospace font

% ===================================================================
% BIBLIOGRAPHY AND CITATIONS
% ===================================================================
% Modern citation and bibliography system (replaces older BibTeX)

\usepackage[
    backend = biber,            % Use biber backend (more powerful than BibTeX)
    language = auto,            % Auto-detect language from babel settings
    style = numeric,            % Citation style: [1], [2], [3] (change to 'alphabetic' for [Doe21])
    sorting = none,             % Don't sort references (appear in order of first citation)
    sortcites = true,           % Sort multiple citations: [1,2,3] instead of [3,1,2]
    block = ragged,             % Allow ragged right alignment in bibliography
    hyperref = true,            % Make citations clickable links
    bibencoding = auto,         % Auto-detect bibliography file encoding
    giveninits = true,          % Use initials for first names (J. Doe instead of John Doe)
    doi=false,                  % Don't show DOI numbers (set to true if needed)
    isbn=false,                 % Don't show ISBN numbers (set to true if needed)
    alldates=short              % Use short date format (2023 instead of 2023-01-01)
]{biblatex}

% Load your bibliography file (where all your references are stored)
\addbibresource{appendix/sources.bib}

% Fine-tune URL line breaking in bibliography
% These settings prevent awkward line breaks in long URLs
\setcounter{biburlnumpenalty}{3000}     % Penalty for breaking at numbers
\setcounter{biburlucpenalty}{6000}      % Penalty for breaking at uppercase letters
\setcounter{biburllcpenalty}{9000}      % Penalty for breaking at lowercase letters

% ===================================================================
% PACKAGE LOADING COMPLETE
% ===================================================================
% All packages are now loaded. The document is ready for content!
% 
% TROUBLESHOOTING TIPS:
% - If you get "Package not found" errors, you may need to install missing packages
% - If you get conflicting package errors, check the loading order
% - Some packages have options that must match your document needs
% ===================================================================
